\newpage\nuluj
\pagestyle{kantorowicz}

\logo

\nadpis{Judges as Fiscal Activists:\\Can Constitutional Review Shape Public Finance?}

\autor{Jaros�aw Kantorowicz\pozn{European Doctorate of Law and Economics (EDLE). The University of Hamburg, Institute of Law and Economics, Johnsallee 35, D-20148 Hamburg, Germany. E-mail: jaroslaw.kantorowicz@edle-phd.eu.}}
\linka{3ex}

\cast{Abstract}
The judicialization of politics, or alternatively, politization of the judiciary has been much discussed over the last twenty years. Despite this, the way judges influence fiscal policy outcomes remains, to a~large extent, unexplored. This paper attempts, at least partially, to fill this research gap. A~judicial (constitutional) review constitutes the central element of the current analysis since it is considered as a~key institutional device through which Constitutional (Supreme) Courts intervene in politics, including public finance. Specifically, this paper seeks to investigate empirically whether there is any systematic pattern according to which judges executing judicial review shape fiscal outcomes. The conceptual framework is based on the strategic interaction model and the assumption that the Constitutional Courts reflect public opinion (i.e. the Court as a~majoritarian institution). Some preliminary results for a~panel of 24 EU countries in the period 1995--2005 suggest that a~strong judicial review correlates with a~smaller size of government, measured as government income to GDP.\\

\cast{Keywords}
Public Finance, Size of Government, Constitutional Court, Judicial Review
\linka{4ex}

\podnadpis{Introduction}
The general observation in research covering the topic of political science and law is that policy-making becomes `judicialized' (e.g. Stone Sweet, 2001, Garoupa, 2011, Garoupa and Ginsburg, 2012). According to Vallinder (1995), the judicialization of politics refers to the shifting of decision-making powers from the legislature and the executive to the courts. To a~large extent, if not exclusively, the judicialization of politics is due to the presence of Constitutional (Supreme) Courts and especially judicial (constitutional) review.\pozn{In this paper the terms judicial and constitutional review are used interchangeably. This is despite the slight difference between these two notions. While constitutional review might be pursued by any institutional body, judicial review is strictly conducted by the judiciary.} Their presence triggers constitutional adjudication, which according to Stone Sweet (2007) constitutes the lawmaking process. 

In the Kelsenian tradition, a~Constitutional Court is explicitly designed to intervene in politics. This approach views the Court as a~\fntit{negative legislator}\hskip 1pt\pozn{Stone Sweet (2000) goes even further when referring to the Constitutional Court as a \fntit{specialized legislative chamber} able to reject legislative statues.} able to reject a~law from an \fntit{ex ante} perspective\pozn{An \fntit{ex ante} judicial review is performed only in an abstract form, i.e. the Court reviews new provisions without reference to a~specific case in which provisions are applied (abstract review).}, thus executing the right of an abstract review (Kelsen 1942). In fact, the role of the current Constitutional Courts is much broader and their competencies go far beyond the \fntit{negative legislator} concept. For instance, the Courts are also able to strike down the law after its promulgation\pozn{An \fntit{ex post} judicial review can be launched in an abstract or concrete form. The latter is initiated by the Court in connection with a~specific case (concrete review).} (e.g. a~concrete review) and to impose statutory interpretation on the ordinary courts. Additionally, the Courts might be engaged in the lawmaking process (e.g. Slovenia), the verification of the legality of elections (e.g. Lithuania) and the legalization of political parties (e.g. Bulgaria). Interestingly, some of these ancillary duties do not even rely on the interpretation of the constitutional text (Ginsburg and Elkins, 2009). This indicates that judicial power expands far beyond its traditional domain.

The underlying inquiry relevant for the current paper is whether this overall tendency of the judicialization of policy extends to the area of public finance. Hitherto, the mainstream literature on institutional public finance\pozn{Institutional public finance literature consists of three strands, i.e. literature on numerical rules, procedural rules and political institutions. The former refers to James Buchanan's concept of rules, such as balanced budget, expenditure, revenue and debt rules. Their sole purpose is to restrain political arbitrariness in public finance. In this vein, once politicians are deprived of full discretion in fiscal policy, fiscal discipline is warranted. The second component of institutional public finance is developed by von Hagen (1992). This approach focuses on the effects of procedures guiding the preparation, adoption and implementation of the state budget. The third element is best conceptualized by Persson and Tabellini (2003). They investigate the impacts of political institutions on different fiscal variables. Their main inquiries pertain therefore to fiscal effects of basic constitutional settings such as (1) the parliamentary versus presidential system, (2) unicameral versus bicameral parliament, (3) proportional versus majoritarian electoral system, (4) unitary versus federal states, and (5) broad versus narrow direct democracy. For an extended overview of the literature on institutional public finance, see Raudla (2010).} largely disregards judges as key institutional players in budgeting. This paper attempts, at least partially, to fill this gap in the research and add to the scarce literature investigating the role of judges in the area of public finance. In this paper, it is conjectured that, by executing constitutional review, the independent judiciary\pozn{Judicial independence means that the judges' decisions and rulings are not influenced by political pressure (Hayo and Voigt, 2007).} might have an impact on fiscal policy outcomes such as, for instance, on general government revenue. Whether this impact is systematic (e.g. consistently leading to lower revenue) will be examined in this paper. 

%\clearpage\newpage

Overall, this paper presents a~positive analysis. To be precise, its main purpose is to examine empirically whether any systematic way exists in which the judicial review shapes fiscal policy outcomes. The general hypotheses are derived based on the strategic interaction model and the assumption that the Constitutional Courts seek public support and reflect general public opinion (Vanberg, 2005). Hence, this study relies on the literature strand relating to the Constitutional Court as a~majoritarian institution. The preliminary empirical investigation with respect to the size of government is pursued for panel data of 24 European Union (EU) member states over the period 1995--2005.\pozn{Cyprus, the Czech Republic, Estonia, Hungary, Latvia, Lithuania, Poland, the Slovak Republic and Slovenia entered the EU in 2004. Bulgaria and Romania accessed it in 2007 and therefore were not EU members in the period under consideration. The remaining countries, i.e. Austria, Belgium, Denmark, France, Germany, Greece, Italy, Ireland, Luxembourg, the Netherlands, Portugal, Spain and Sweden, were members of the EU in the whole period under investigation.} The results included in this paper show that a~larger degree of constitutional review correlates with smaller governments, measured as general government revenue relative to GDP. Consequently, countries where judges are equipped with more constraining judicial review tend to experience lower government revenue compared to GDP. 

Constitutional review is the main legal device through which the Court factually intervenes in the political sphere and thereby influences public finance outcomes (see Section 2). In this study, judicial (constitutional) review is defined as the ability of the Court or other judicial body to verify whether the laws and regulations enacted by the legislature are in line with the constitutional provisions and in accordance with procedural requirements (Ginsburg, 2008). The crucial consequence of the review, at least from a~theoretical standpoint, is that laws and regulations which fail to comply with the constitutional provisions are invalidated or revised by the legislature in line with the Court's opinion. The rationale behind the ability of judges to review the legislation is given by Landes and Posner (1975). According to them, the review serves as a~means of dealing with the commitment problem, which emerges as a~result of the incompleteness of constitutional contracting.\pozn{Constitutions, as any other contracts, are incomplete, i.e. they do not regulate for all possible contingencies. This is due to the insurmountable transaction costs hypothetically involved in designing a~complete contract (Sch�fer and Ott, 2004).} Consequently, the Constitutional Court, or another institution equipped with judicial review power, functions as an external dispute resolution mechanism between citizens and the state authorities, deterring and correcting the latter for any abuse of power. This is analogous to private contracting, where the existence of formal enforcement mechanisms incentivize parties to comply with the contract, despite its incompleteness.

Despite the possible relevance, this study does not comprise a~historical overview of the creation and implementation of constitutional (judicial) review around the world. Comprehensive surveys on these topics can be found, for instance, in Deener (1952), and Stone Sweet (2000). Similarly, the current analysis does not relate to the debate on the democratic or, alternatively, undemocratic foundation of constitutional review. For this discussion, see Waldron (2006), Fallon (2008) and Tushnet (2010). Further, this study does not refer to the problem of delimitation of jurisdictions, including conflicts of competence between the Constitutional and the Supreme Courts. Garlicki (2007) offers an in-depth analysis of this issue.

The remainder of this paper is structured as follows. Section 2 provides the motivation and anecdotal evidence on Courts influencing fiscal policy outcomes. Section 3, in turn, comprises a~brief survey of literature which is relevant to the topic at hand. Section 4 discusses the possible transmission channel between judicial review and fiscal policy outcomes. In addition, it proposes hypotheses for empirical investigation. Section 5 describes the data and Section 6 sketches the econometric strategy to be applied in testing for selected hypothesis. Section 7 discusses some preliminary results and Section 8 enumerates basic limitations of the study.  Lastly, Section 9 concludes.

\podnadpis{Motivation}
The ruling of the German Constitutional Court is probably the most up-to-date example of judicial fiscal activism\pozn{Judicial fiscal activism is referred to as any decision and ruling of the judges which results in higher or lower taxation and, respectively, higher or lower public spending. Although, in theory the judges should be allowed only to reject or abolish the legislation, in fact judicial fiscal activism might also concern lawmaking, which is occasionally costly for the budget (see the Colombian case further in the current section).}. In mid-September 2012, the Federal Constitutional Court in Karlsruhe rejected a~lawsuit which questioned the German ratification of the European Stability Mechanism (ESM)\pozn{The ESM is the successor of the European Financial Stability Facility. In short, the ESM is a~rescue mechanism granting loans to Eurozone Member States. It also aims to provide precautionary financial assistance, purchasing bonds of Eurozone countries on primary and secondary markets and recapitalizing financial institutions (see http://www.european-council.europa.eu/homepage/highlights/european-stability-mechanism-treaty-signed accessed on October 22, 2012).} and fiscal compact\pozn{The fiscal compact (officially, the Treaty on Stability, Coordination and Governance in the Economic and Monetary Union) constitutes a~fiscal constitution for the Eurozone Member States. It provides, among other things, that the limit of the annual structural deficit should not exceed 0.5\% of the GDP. However, occasionally, when debt-to-GDP is significantly lower than 60\%, the state may run a~structural deficit of 1\% of GDP. As the Treaty instructs, the new fiscal rule shall be enshrined in the highest national statutory provision, i.e. preferably in the Constitution (see http://europeancouncil.europa.eu/eurozone-governance/treaty-on-stability?lang=nl accessed on October 22, 2012).}. The decision to reject the lawsuit allowed the inclusion of the ESM and the fiscal compact in the German legislation. It is worth emphasizing that such a~ruling was indispensable for both of these euro-rescue policies to start operating. However, the Court's role did not terminate with a~mere `no' decision against the lawsuit. Throughout the decision, the judges set a~range of formal conditions regarding German participation in the ESM. Particularly, they discretionally capped at 190 billion euro the German contribution to the ESM and required the consent of the German parliament to increase this amount.\pozn{See http://www.bverfg.de/en/decisions/rs20120912\_2bvr139012en.html (accessed on November 5, 2012).} Thus, on the one hand, the judges approved German participation in the rescue policies, which was potentially important to restore confidence in the Eurozone in light of the debt crisis. On the other hand, they limited the extent of German contribution and imposed rigidities under which the state could augment the scope of the funding. Inevitably, both aspects of this decision affect the budgetary process and consequently curb German public finance.\pozn{In 1995, another important decision of the German Constitutional Court restrained the fiscal maneuvering of the federal government by declaring unconstitutional the federal property tax (see http://www.economist.com/node/21562237 accessed on October\;21, 2012). Moreover, in 2008, the Constitutional Court in Karlsruhe invalidated the amendment of the commuter tax allowance. The amendment abolished the deduction of costs for travelling to the workplace up to 20 kilometers. In the wake of the Court's decision, the government was obliged to return overpaid taxes, i.e. an amount of roughly eight billion euro (see http://www.bverfg.de/pressemitteilungen/bvg08-103en.html accessed on November 25, 2012). Finally, in 2010, the Constitutional Court ruled unconstitutional the law imposing higher inheritance taxes on homosexual couples as compared to heterosexual couples (see http://www.bverfg.de/en/decisions/rs20100721\_1bvr061107en.html accessed on November 18, 2012).} 

Clearly, Germany is not the sole country where judges played an important role in the fiscal policy area. For instance, in July 2012, the Portuguese Constitutional Court declared the austerity plan, which was launched by the federal government as a~measure to contend with fiscal crisis, unconstitutional. The judges opposed a~crucial part of the plan which suggested limiting an extra holiday and Christmas pay for public sector workers. In the Court's opinion, the deficit-cutting program infringed the principles of equity. According to the judges, the cost of fiscal consolidation was unequally distributed among public and private sectors, imposing too heavy burdens on the former.\pozn{See http://www.bbc.co.uk/news/world-europe-18732184 (accessed on October 20, 2012).} It seems plausible to conjecture that, as a~result of this decision, the fiscal tightening program in Portugal was delayed. Similarly to Portugal, some austerity measures were ruled unconstitutional, for instance, in the Czech Republic (concerning the reduction of judges' salaries)\pozn{See http://praguemonitor.com/2010/09/13/constitutional-court-cancels-reduction-judges-salaries (accessed on October 20, 2012).} and in Romania (regarding cuts in pensions)\pozn{See http://opportunity.ro/en/content/romania\%E2\%80\%99s-parliament-eliminates-unconstitutional-provisi\-ons-austerity-plan (accessed on October 20, 2012).}. The list of countries where the fiscal activism of the Constitutional Court has been observed in recent years could be easily extended by including France\pozn{According to Stone (1995), over the period 1974--1990, all annual budget laws in France were referred to the Constitutional Council. While the budget laws were not necessarily rejected by the Council, the fact that they were subject to the review excluded the radical tendencies and promoted the status quo (Favoreu, 1986).} (cancelation of the carbon tax)\pozn{See http://news.bbc.co.uk/2/hi/europe/8434505.stm (accessed on October\;21, 2012).} and Italy (annulation of the luxury tax)\pozn{See http://www.mondaq.com/x/186704/Aviation/The+Italian+luxury+tax+on (accessed on October~21, 2012).}. Overall, each of these decisions by Constitutional Courts limited, at least to some extent, the scope of government discretionary action in the area of fiscal policy.

However, Hungary constitutes another pivotal case. According to the new Constitution of 2011\pozn{Although the new Hungarian Constitution was promulgated in 2011, it entered into force on January\;1, 2012 (see http://www.europeanvoice.com/article/2012/january/commission-raises-concerns-about-hungary-s-constitution/73086.aspx accessed on November\;18, 2012).}, the Hungarian Constitutional Court is explicitly excluded from ruling on issues related to budget and taxation. This limitation will be upheld until public debt drops to 50\% of GDP from the current level of roughly 80\% of GDP.\pozn{See article 37(4) of the new Hungarian Constitution (see http://www.mkab.hu/rules/fundamental-law accessed on November 5, 2012).} Interestingly, prior to this constitutional reform, the Hungarian Constitutional Court was the most activist court in the Central and Eastern Europe\pozn{The Polish Constitutional Tribunal is also known for its fiscal activism. In 1996, judges abolished the suspension of the indexation of pensions and, in 1997, they struck down a~number of provisions of the 1991 Tax Statue (Sadurski, 2002).}, if not the world, including activism in the fiscal policy area (Sadurski, 2002). For instance, in 1995 judges ruled unconstitutional 26 provisions of the austerity plan, such as the abolishment of social entitlements (e.g. sick leave benefits and family allowances), cancelation of pension plans and staffing cuts in higher education (Schwartz, 2002). According to the Hungarian Ministry of Finance, those decisions were equivalent to 20--30\% of the value of the entire austerity program. Drawing from past experience and more recent Court's decisions, which once again were mostly unfavorable to the budget consolidation\pozn{The Court's cancelation of the retroactive effect of a~98\% tax on some severance pay for public sector employees initiated the government's decision to abolish the Court's authority in Hungary (see http://www.bbj.hu/politics/constitutional-court-annuls-retroactive-effect-of-severance-pay-tax\_57624 accessed on October 20, 2012).}, the \fntit{Fidesz-KDNP} coalition government\pozn{\fntit{Fidesz} is a~main party of the Hungarian Parliament. It holds roughly 59\% of the seats in the lower chamber. KDNP, the collation partner, occupies another 10\% of the seats. Together this gives a~majority of more than $2/3$, which is required for constitutional amendments. For the special situation of the government coalition and the \fntit{Fidesz} party particularly, see http://www.reuters.com/article/2010/04/12/us-hungary-election-idUSTRE63A1GE20100412 (accessed on November 18, 2012).} decided to curtail the Court's activity. This controversial move could be pursued because the conservative \fntit{Fidesz-KDNP} coalition had attained the qualified majority of $2/3$ of parliamentarian votes. Due to the restricted role of the Court, the government was able to regain full discretion in fiscal policy and launch a~fiscal adjustment process.\pozn{To some extent, the Hungarian case is analogous to the court-packing attempt announced by Franklin Delano Roosevelt in 1937. The direct reason for this plan was the fact that, in 1935, the U.S. Supreme Court struck down a~large segment of the New Deal and, in 1936, it declared unconstitutional some further parts of it (including the Agricultural Adjustment Act, the Guffey Coal Act and the Municipal Bankruptcy Act). According to the judges' statements, the national government had no constitutional capacity to take action for economic recovery in the aftermath of the Great Depression. Although the court-packing attempt failed, the Court eventually ruled in favor of the New Deal legislation. Eventually, a~nexus of circumstances led to the Court's agreement. One of those was the retirement of Justice Willis Van Devanter, who was an intellectual leader of the Court's conservative wing. However, growing public sympathy towards the New Deal program also played an important role in changing judges' attitudes (Caldeira, 1987).} The latter was, to a~large extent, revenue driven, i.e. fiscal consolidation was achieved through an increase in taxes\pozn{For tax increases in Hungary, see e.g. http://www.bloomberg.com/news/2010-07-22/hungarian-lawmakers-to-approve-brutal-bank-tax-in-defiance-of-imf-eu.html (accessed on November 8, 2012).} and the nationalization of the private-pension fund\pozn{For the private pension-fund nationalization, see e.g. http://www.bloomberg.com/news/2010-1125/hungary-follows-argentina-in-pension-fund-ultimatum-nightmare-for-some.html (accessed on November 8, 2012).}. The government's decision to diminish the importance of the Constitutional Court remains controversial and is perceived as a~violation of the rule of law and standards of democracy.\pozn{See http://www.nytimes.com/2011/04/19/world/europe/19iht-hungary19.html?\_r=3\& (accessed on October 20, 2012).} The Hungarian case demonstrates the dynamics in the relationship between the Constitutional Court and incumbents. Whenever the Court is too activist in restraining political fiscal discretion, governors might intend to undermine its role, once they are able to amend the constitution.

Although this paper focuses on Europe, it is certainly not the only place where the Constitutional Court's decisions interfere with reforms of budgetary allocations and tax collection. It turns out, for instance, that judges are key players in the budget process in Colombia. A~decade ago, the Colombian Constitutional Court declared unconstitutional a~law imposing a~2\% VAT on items which were previously free of taxes and another law which aimed at reducing pensions (Eslava, 2006). Interestingly, constitutional judges in Colombia not only annul laws but are active in lawmaking and the imposing of revised spending on the executive. For example, in 2004, the Constitutional Court, dealing with internal displacement problems, determined that the rights of families, forced to leave their premises as a~consequence of the state's inner conflict, were violated. In the aftermath of the Court's decision, the government was obliged to develop a~compensation mechanism for displaced individuals (C�rdenas et al., 2009). 

Judicial fiscal activism is also present in Israel. In a~very recent case from 2012, the Israeli High Court of Justice revoked a~provision in the Income Tax Ordinance. The latter provision provided tax benefits for certain settlements. According to the Court, the methods by which the settlements were selected for receiving benefits impeded the right to equality and unconstitutionally discriminated between proximate and substantively similar areas. Besides declaring void a~provision which provided tax benefits for some Jewish settlements, the Court instructed the legislator to include three Arabic areas for tax reimbursements.\pozn{HCJ (8300/02) \fntit{Gadban Nasar and the Local Council of Mazraa vs. State of Israel} and others (see, in Hebrew, http://elyon1.court.gov.il/files/02/000/083/n48/02083000.n48.htm accessed on November 19, 2012).}     

All these cases demonstrate that the review of the constitutionality of the law is crucial for the Courts to influence public finance outcomes. One may therefore conjecture that the judicial (constitutional) review is the main legal device through which the Courts actually intervene in the political domain and thereby affect public finance outcomes. The underlying inquiry as to whether the Courts, through the execution of the judicial review, shape fiscal policy outcomes in a~systematic way, remains unanswered.\\[-2mm]

\podnadpis{Related literature}
Initial interest in judiciary as a~possible explanatory variable of economic outcomes can be traced back to the mid-1970s.\pozn{Landes and Posner (1975) were among the first to analyze judicial independence from the economic point of view. According to them, independent judiciary is in the interest of the legislature. The reason for this is that the presence of independent judges prolongs the lifespan of legislation, which is desired by certain constituencies (interest groups).} However, the continuing interest in the judicial influence on economic policies began in the early 2000s. For instance, Henisz (2000) investigates how the institutional environment, including an independent judiciary, ensures the credible commitment of the government not to interfere with the private property rights. The latter seems crucial, since potential governmental attenuation or the expropriation of private property rights disincentives capital accumulation, which is crucial for economic growth. The author presents empirical evidence using panel data of 157 countries over the period 1960--1994. He demonstrates that, by enhancing credibility of commitment, institutional constraints (including independent judiciary) positively affect economic development.

Feld and Voigt (2003), in turn, emphasize that the distinction between \fntit{de jure} and \fntit{de facto} judicial independence is crucial when analyzing the economic consequences of institutions. While \fntit{de jure} independence refers to the mere legal provisions, the \fntit{de facto} measure stands for the actual independence of judges. The latter is approximated, among others, through effective term lengths and the degree to which judicial decisions influence government behavior. The authors do not find support for the hypothesis that \fntit{de jure} judicial independence positively influences the growth of real GDP per capita. It is contrary to the \fntit{de facto} judicial independence, for which positive impact on rate of economic development is detected. In their empirical investigation, Feld and Voigt rely on cross-sectional data from 56 countries, for which information on the \fntit{de jure} and the \fntit{de facto} judiciary independence is found.

In Feld and Voigt (2006), the authors extend their cross-sectional analysis to 73 countries. The positive impact of the \fntit{de facto} judicial independence on the rate of economic growth is sustained.\pozn{A~completion of judges' term, marginal changes to the number of judiciaries and competitive wages of judges are the most crucial components of the \fntit{de facto} judicial independence conducive to economic growth.} In addition, it is shown that some of the components of the \fntit{de jure} judicial independence positively influence economic development.\pozn{Two components of the \fntit{de jure} judicial independence which are positively associated with economic growth are accessibility to the lower court and the length of the appointment term for the highest court judges.} However, the Constitutional or the Supreme Court's power of judicial review turns out to be negatively correlated with real GDP growth.

In a~less rigorous fashion, the positive influence of judiciary independence on economic growth is reported by Wittrup (2010). The author's empirical evidence is based on cross-sectional data from 95 countries. Some of the variables in the model are averaged over the period 1980--2003. The key variable of interest, judicial independence, relies on measures from a~perception survey (the Executive Opinion Survey) prepared by the World Economic Forum\pozn{For the methodology of the survey, see https://wefsurvey.org/index.php?sid=28226\&lang=en\&intro=0 (accessed on November 10, 2012).}. According to the author, this perception measure serves as a~proxy for \fntit{de facto} judicial independence.

Contrary to the aforementioned studies, which seek a~link between the judiciary and economic growth, La Porta et al. (2004) discuss how judicial independence and constitutional review ensure economic and political freedom. Their cross-country empirical investigation, encompassing 71 countries, indicates that more judicial independence is correlated with both larger economic and political freedoms. Constitutional review is, in turn, significantly related to economic freedom only when the latter is approximated by the property rights index.

The abovementioned studies are important inasmuch as they investigate the relationship between judiciary and economic phenomena such as economic growth and freedom. The central inquiry of this paper relates, however, to the correlation between the judicial review and public finance outcomes. The relevant literature on the topic at hand is very limited. To the author's knowledge, there are only five academic works which treat judicial independence and$/$or judicial review as explanatory variables of some of the fiscal policy outcomes. 

First, in the theoretical model, Padovano et al. (2003) propose that an independent judiciary enhances political accountability in democratic systems. This increased accountability leads, in turn, to higher social welfare, since the provision of public good is maximized (public revenue is not extracted through political rent seeking). Contrary to this, accommodating judiciary, which colludes with other government branches, results in lower social welfare. The latter is due to the extraction of tax revenue by rent-seeking governmental branches.

Second, Vaubel (1996) suggests that the existence of the Constitutional Court is an important factor of expenditure centralization. According to the author, judges of the Constitutional Court have centralist preferences, i.e. they tend to favor and strengthen central institutions inasmuch as centralization enhances their prestige and influence. Consequently, Constitutional Court judges are interested in transferring competences from the province to the federal level, since only then can the Court be in charge of interinstitutional disputes, which were previously ruled at the province level. The empirical evidence is given for cross-sectional data from 50 countries in 1989--1991. In his later study, Vaubel (2009) confirms the previous results and further expands analysis on this area of research. First, the author describes that centralization is larger in those countries where judges of the higher court enjoy independence from the other governmental branches. Second, the author states that centralization is larger when the barriers to constitutional amendment are significant. The dataset for cross-sectional empirical investigation encompasses 42 economies between the years 2001 and 2004.

Third, Tridimas (2005) shows theoretically and empirically that a~stronger judicial review and judicial independence are associated with a~relatively lower size of government. The latter is measured by central government revenue to GDP. The existence of the Constitutional Court, with the ability to strike down legislation, introduces political uncertainty and limits government discretion in levying taxes. It is assumed that the government is automatically corrected by the Court for setting too burdensome taxation and promulgating tax provisions which are incompatible with the constitution. Consequently, the presence of the Court leads to a~decrease in the size of politically optimum redistributive measures. The theoretical underpinnings are supported by the cross-sectional empirical investigation. The latter is conducted for a~sample of 52 countries. 

In addition, Eslava (2006) presents suggestive empirical evidence that judicial activism in fiscal policy results in a~larger public deficit. Her empirical investigation encompasses 23 South and Central American countries in the period 1996--2003. According to Eslava, the larger public deficit is due to the delay in fiscal reform. As she suggests, fiscal consolidation costs usually fall disproportionately on certain social groups. These groups, perceiving that their rights were violated (as compared to other groups, which stayed untouched), are incentivized to take legal action, i.e. file the case with the Constitutional Court, and thus hamper the introduction of austerity measures. Usually, the benefits of fiscal tightening exceed its costs. The benefits are, however, internalized by all citizens and, consequently, are small and difficult to quantify for an individual. This leads to a~situation where individuals who benefit from the fiscal consolidation do not organize themselves to secure it. Due to this collective action problem, the Court rules exclusively on cases brought by organized social groups, who lose from fiscal consolidation and thus have an interest in requesting the Court to strike down the austerity measures. As a~result, this asymmetry of ruling results in larger deficits and a~delay in the fiscal adjustment. 

Lastly, Raudla (2011) demonstrates the impact of the Constitutional Court on the tax system, based on the case study of Estonia. In-depth analysis of the Estonian Constitutional Court's decisions on taxation allows the author to conclude that Courts in transitional economies are willing to impose costly judgments for the public budget. Radula conjectures that, in transformation countries, newly created Constitutional Courts seek to establish reputation and gain public support. As has been stressed, this reputation may be gained by supporting the interest and rights of the taxpayers.\\

\podnadpis{General hypotheses}
This paper attempts to empirically examine whether the judiciary, through the execution of the constitutional review, shapes public finance outcomes in a~systematic way. In constructing theoretical underpinnings for the empirical test, it is necessary to presume which incentives of the Constitutional Court's judges prevail. Hitherto, the literature investigating the judicial behaviors in ruling the constitutionality of law is not unanimous. On the one hand, the \fntit{attitudinal approach} perceives judges as unconstrained players who rule on cases based on their political preferences (Segal and Spaeth, 2002). On the other hand, in the \fntit{strategic interaction models}, judges' decisions respond to the changing political circumstances and adjust to potential reactions of the other actors, such as the legislators and electorate (Gely and Spiller, 1992). 

In this study, the hypotheses regarding the potential influence of the constitutional review on different variables of public finance are derived based on the strategic interaction model presented by Vanberg (2005). A~strong assumption is made that judges seeking for the authority and the integrity of the Constitutional Court reflect general public preferences in their decisions. 

Constitutional Courts do not function in an institutional or political vacuum (Gely and Spiller, 1990). Contrary to this, Courts are clearly involved in strategic interactions with other actors. These are interactions between the Court and the legislature, on the one hand, and between the Court and the general public, on the other. The presence of the Constitutional Court leads to the adjustment of the decisions made by the legislators, who are aware that over-radical or unlawful provisions might be struck down. This ability of the legislature to assess the Court's reaction and abandon unconstitutional legislation is defined as autolimitation (Stone Sweet, 2000). Public support, in turn, which is the main scope of analysis in the current paper, seems to be crucial to legitimatize the Court's decisions and strengthen the execution of its veto power. 

As claimed by Mishler and Sheehan (1993), the Court seeking authority and recognition is reluctant to deviate in its ruling from public opinion. This is reflected in Justice Frankfurter's famous expression that ``the Courts' authority -- possessed of neither the purse nor the sword -- ultimately rests on sustained public confidence in its moral sanction'' (in \fntit{Baker v. Carr}\hskip 1pt\pozn{The case can be retrieved from https://supreme.justia.com/cases/federal/us/369/186/case.html (accessed on November 27, 2012).}). Similarly, McGuire and Stimson (2004) claim that judges lack institutional capacities to ensure the full effect of their rulings. Consequently, without public support, their preferred outcomes might be rejected and the Court's legitimacy undermined. Also, Posner (2008) declares that ``the usual external constraints on judicial discretion are severely attenuated except for public opinion'' (p. 274).

In the same vein, Vanberg (2005) considers public support as a~crucial \fntit{judicial resource}. As claimed, public support strengthens the \fntit{de facto} power of the Constitutional Court and the judicial review in the net of strategic interaction with other political players. Although judges are policy motivated, they are also likely to have institutional concerns. Non-compliance with the judges' decision by the legislature is possibly costly for the Court, since it undermines its authority by challenging its role in the policymaking. Consequently, a~successful evasion of the Court's ruling weakens its position \fntit{vis-\`a-vis} other political bodies.

Crucial in this setting is the potential \fntit{problem of compliance} which emerges due to the fact that the Court's decisions are not self-enforcing. Implementation of the Court's ruling involves the cooperation of other actors (especially legislators) who may not wish to comply with a~certain decision of the Court. It requires the legislature to act according to the Court's ruling and revise those provisions which were declared unconstitutional. The legislators' incentives in deciding how to respond to the Court's decision are essential. Without any pressure to enforce the decision, legislators are incentivized to act in an opportunistic fashion and evade the Court's ruling. It cannot be excluded that, even if the law is annulled, legislators might be tempted to re-enact unconstitutional legislation and thus circumvent the Court's decision (Bossuyt, 2008). As claimed by Vanberg (2005), a~key mechanism that creates pressure for legislators is the possibility of a~negative public backlash in the event of non-compliance with the judicial decision. Public support for the Court or its specific ruling is therefore perceived as an \fntit{enforcement mechanism} and a~driving force for judicial decisions to be implemented. In a~nutshell, fear of a~potential negative electoral reaction (i.e. reputation loss and negative electoral consequence) incentivizes the legislature and the government to revise the unconstitutional provisions or abstain from the re-introduction of unconstitutional law.\pozn{The empirical investigation of negative electorate consequences as a~result of non-compliance with the law by governors has not been yet pursued. Based on anecdotal evidence, Schauer (2012) proposes a~refined hypothesis of illegal action of governors and the resulting electoral effect. He claims that illegality of policy does not result in a~negative electoral effect if this policy appears to be successful. Contrary to this, illegal action which leads to negative outcomes is likely to result in aggravated negative political and reputational consequences. In the latter situation, politicians are punished for both unsuccessful policy and the breach of the law. Empirical investigation of the underlying hypotheses constitutes a~promising avenue of research.}

There are two types of public support the Constitutional Court might enjoy, namely, specific and diffuse support. Specific support refers to a~particular decision of the Court, i.e. one does not necessarily support the Court as such, but is vastly interested in its particular rulings. Diffuse support, in turn, is associated with common support for the impartial Court as an institution. Diffuse support means therefore that one might not agree with a~specific ruling of the Court but will still respect it as an independent constitutional safeguard. According to Vanberg (2005), in time, specific support converts into diffuse support for the Court as an institution. 

Based on this discussion, it is assumed therefore that the level of diffuse public support the Court enjoys can be enhanced via a~specific ruling which is in line with the public will. According to Gibson and Caldeira (2003) and also Raudla (2010), the specific ruling in line with the public mood occurs in particular shortly after the Court's establishment. At the early stage, the new institution seeks legitimacy, reputation and recognition \fntit{vis-\`a-vis} other institutional and political actors by attracting public support. Similarly, when the Court already enjoys diffuse public support, the judges are aware that this valuable resource can rapidly be wasted if systematically unpopular and unsatisfactory decisions against prevailing public attitudes are made. Overall, public opinion and preferences impose boundaries on the Court's actions and can be understood as driving forces of the judicial ruling. Empirical and anecdotal evidence that the Courts are indeed sensitive to prevalent public moods can be found, for instance, in Mishler and Sheehan (1993, 1994, 1996), Link (1995), Stimpson et al. (1995), Flemming et al. (1997), Dotan (1998), Volcansek (2000), McGuire and Stimson (2004), Vanberg (2005), Friedman (2005), Giles et al. (2008), Posner (2008), Ura and Wohlfarth, and also Casillas et al. (2011).

Although most studies use empirical evidence from the U.S. Supreme Court\pozn{Empirical investigation of the U.S. Supreme Court reflecting public opinion is possible due to the availability of public mood indicators, i.e. the domestic policy mood index (see, e.g., McGuire and Stimson, 2004).}, Dotan (1998) shows in an anecdotal fashion that the Israeli High Court of Justice often follows the opinion of the majority of the electorate. Those are especially pro-majoritarian decisions limiting Orthodox Jewish law. More importantly, Volcansek (2000) and Vanberg (2005) demonstrate that Kelsenian type Courts are also responsive to public opinion. In interviews conducted by Vanberg (2005), judges tended to state that ``the court does not take an opinion poll, but public attitudes does play a~role'' or ``they [the judges] are well aware of the public mood\;\dots the public mood is very important for the judges'' (p. 128--129). Moreover, judges admitted that they cannot frequently allow themselves to rule against prevailing attitudes. Another example of judges tending to reflect public opinion can be found in Poland. In 2010, the Polish Constitutional Tribunal refrained from ruling unconstitutional the law reducing the pensions of the former agents of the secret service operating under socialism.\pozn{Decision K~6/09 of the Polish Constitutional Tribunal (for decision, see, in Polish, http://www.trybu\-nal.gov.pl/OTK/otk.htm accessed on November 28, 2012).} The law cut the retirement benefits of more than 40,000 pensioners by an average of 20\%. The case was highly controversial, as it infringed on the protection of acquired rights. The Court's decision was, however, in line with broad public preferences (i.e. 58\% of respondents supported this initiative).\pozn{See, in Polish http://wiadomosci.gazeta.pl/wiadomosci/1,114873,4129355.html (accessed on November 28, 2012).}

As shown heretofore, public support should be crucial for the Court. This support shields the Court's authority (strengthening the enforcement of the decision) and its institutional integrity. Consequently, in the context of fiscal policy, a~Court targeting to gain or maintain public support should represent the will and preference of a~public which is usually hostile towards any tax increase and abolishment of tax exemption. It is quite intuitive that people universally prefer lower rather than higher tax burdens. For instance, Hansen (1998) shows through a~survey that the general public in the U.S. largely opposes the increase of taxes in order to cut the budget deficit.\pozn{For instance, 75.3\% of respondents oppose increases in taxes in order to cut the budget deficit. Moreover, 64.6\% respondents stand against tax increase in order to augment spending on domestic programs, such as health care, education and road infrastructure. Lastly, 86.8\% of respondents object to any increase of taxes to finance larger spending on national defense. The survey was conducted on 486 Americans (Hansen, 2008).} The Eurobarometer (1998) poll, in turn, shows that, although EU citizens are willing to increase public health spending, the vast majority are reluctant to increase taxes to this end. The preferred policy is to cut other spending or to finance it by other means. 

In addition, if it is assumed that the electorate is subject to fiscal illusion\pozn{Although the concept of fiscal illusion could be traced back to John Stuart Mill and Vilfred Pareto, it was Amilcare Puviani who emphasized its importance in the early 20th century. In a~nutshell, fiscal illusion refers to the source of government revenue, which is unidentifiable by the electorate, e.g. debt financing. Inasmuch as the electorate is unaware of the revenue source for the expenditure, it does not perceive its burden (Mueller 2003). The debt financing seems less costly than tax financing, due to a~failure by the electorate to entirely account for the future tax liability caused by debt re-financing. Some empirical evidence of fiscal illusion can be found in Wagner (1976), and also in Pommerehne and Schneider (1978).} (\fntit{`more for less'} paradox\pozn{The \fntit{`more for less'} paradox relies on the concept of the fiscal illusion. This relates to the fact that people simultaneously desire more governmental services and tax reduction (Welch 1985). An alternative explanation of people's unwillingness to accept an increase of taxes in order to raise social spending could be offered by the behavioral law and economics approach, i.e. bounded willpower. According to the latter, people tend to act against their long-term interests (Jolls et al., 1998). For example, individuals prefer to spend money today rather than saving it for the future. Therefore, tax payers might prefer lower taxes in the present instead of paying for a~policy, e.g. social security, which may benefit them in the future.}), then the judiciary should not only promote low taxation\pozn{Of course, this is not to say that Constitutional Courts strike down all legislation which increases taxation. The legislator surely has the power and legitimacy to increase taxation. Consequently, if the legislation which increases tax is in line with constitutional provisions it should be deemed lawful. That is why it is to be declared as unconstitutional such law needs, for instance, which discriminate against one group of people against another. It is presumed that Courts might be more prone to rule the case unconstitutional when the discriminated group is large. Moreover, judges do not act completely irrationally.} but also larger public spending. This, in turn, may lead to systematic fiscal imbalance and the accumulation of public debt. It might further be the case that the Court tends to oppose cuts in social spending, knowing that these reductions are unpopular with the public. For evidence that social spending cuts bring social discontent, see De Vries and Hobolt (2012). Moreover, judges might be resistant to accepting cuts in capital and development spending, since they also enjoy large public support. For evidence on the latter point, see Schuknecht (1994) and Brender (2003).

These, however, would not be correct conjectures regarding expenditure and deficit if alternative assumptions about public fiscal perceptions are true. For instance, Peltzman (1992) claims that U.S. voters should be defined as fiscal conservatives, since they systematically punish governments for the growth of public expenditure. In the same vein, fiscally conservative attitudes of the public are presented by Alesina et al. (1998) and Alesina et al. (2011). Despite the conventional wisdom that fiscal consolidation is a~politically unattractive venture, these studies do not find empirical evidence for this statement. In OECD countries, incumbent governments radically pursuing fiscal adjustments are not systematically worse off than their counterparts which abstain from fiscal tightening. 

Although only in an anecdotal fashion, one may claim that in Switzerland the public is to a~large extent fiscally conservative. This conjecture is derived based on circumstances from the 1990s, when the fiscal stance in Switzerland deteriorated and public debt reached the unobserved level of nearly 60\% of GDP. In 1995, a dissatisfied public launched a~popular initiative to restrain the accumulation of public debt. The aim of the initiative was to introduce constitutional rule to preclude the situation of permanent deficits. The rule was to balance expenditure and revenue in a~four-year time horizon (Conseil Federal, 2000). Despite the defeat of the initiative, this example shows that society was concerned about debt accumulation and wanted to use its right to restraint politicians from spending excessively. Eventually, in 1999, the Swiss Ministry of Finance launched a~project on the debt brake, which was in line with the social expectations. The constitutional provision of the rule was approved in referendum in 2001. Overall, 85\% of voters were in favor of it. The referendum turnout was approximately 37\%, which is high for Swiss standards (Kirchg�ssner, 2005).  

Consequently, while there is agreement concerning the fact that people on average prefer lower taxation, there is a~lack of consensus on the popular fiscal perception regarding expenditure and deficit. Contrary to fiscal illusion, fiscal conservatism leads to more constrained public spending and lower public deficit. Instead of arbitrarily choosing the prevailing perception on expenditure and deficit, it seems plausible to assume that in some countries fiscal illusion is the dominant fiscal perception of expenditure and deficit (e.g. Greece and Italy) and in some other states the public is fiscally conservative (e.g. Estonia, Luxembourg). 

As a~result, one can derive hypotheses conditioned to the fiscal perception of the public. They are as follows: 

\fntit{H1}: The presence of the Constitutional Court or another judicial institution with a strong constitutional review power results in smaller general government tax revenue to GDP regardless of public fiscal perceptions.

\fntit{H2}: The presence of the Constitutional Court or another judicial institution with a strong constitutional review power results in smaller general government spending to GDP if public fiscal conservatism prevails.

\fntit{H3}: The presence of the Constitutional Court or another judicial institution with a strong constitutional review power results in larger general government spending to GDP if public fiscal illusion prevails.

\fntit{H4}: The presence of the Constitutional Court or another judicial institution with a strong constitutional review power results in larger deficits if public fiscal illusion prevails.

It is necessary to mention that information on the fiscal perception of the people is hard to collect. For that reason, in this paper, an attempt is given only to test the first of the hypotheses, i.e. H1.

\podnadpis{Data description}
As previously mentioned, the Constitutional Court or other judicial entities intervene in the political system through the execution of power to review the legislation. However, the constraining effect of the constitutional review varies across countries and time. 

The new dataset by Gutmann et al. (2011) allows for cross-country and cross-time comparability of the constraining effects of the constitutional review, which is a~key variable of interest in the current analysis. The measurements refer to the \fntit{de jure} constitutional review, i.e. what are its legal (constitutional) foundations.\pozn{The coding requires a~short explanation. For instance, according to Belgian statutory law, there is the possibility of strong judicial review. However, this country was classified as not having judicial review power at all since the Constitution does not include precise provision on judicial review. The rationale behind this coding (concentrated on constitutional provision) is that the strongest judicial review power is given by the Constitution. While statutory law might be changed by a~simple parliamentarian majority, the Constitution, at a~minimum, requires a~qualified majority to be amended. Hence, judges are presumed to be more activist when the judicial review power is shielded by the Constitution and not ordinary legislation.} Gutmann et al. examine four elements of the constitutional review. First, they consider only the constitutional review which is pursued by the judiciary. However, it does not matter if the judges reviewing the constitutionality of the law are seated in a~Constitutional Court or a~Supreme Court. There\-fore, a~country where judicial review is pursued by the judiciary receives 1, otherwise 0. If at this stage a~country receives 0, it is not subject to further investigation.

Second, analysis is made of who has the power to initiate the judicial review, i.e. whether the authority to file the case for judicial review is given only to political bodies or to the broader public. Gutmann et al. code nine possibilities of initiating the judicial review\pozn{The distinction is made between the following initiators of the judicial review: (1) the head of the state, (2) the head of the government, (3) the government, (4) the first chamber of the parliament, (5) the second chamber of the parliament, (6) both chambers in conjunction, (7) lawyers, (8) the public, (9) the courts.}. The data are normalized on the 0--1 scale, where 1 means that in a~particular country all nine channels of filling the case for the constitutional review are present and 0 where none of the channels are identified.

Third, when the constitutional review can be launched is examined, i.e. before (\fntit{ex ante}) or after (\fntit{ex post}) constitutional review. Gutmann et al. propose to assign 1 to a country where it is possible to initiate the judicial review before and after the promulgation of the law. The value of 0.5, in turn, is assigned to a~country where the judicial review can be launched only \fntit{ex post} and the value of 0, where the law can be reviewed only \fntit{ex ante}.

Finally, the authors investigate the effects of declaring the law as unconstitutional, i.e. it is automatically void or only returned to the legislature for re-consideration. Gutmann et al. assign 1 to a~country where the law is automatically void, and 0.5 to a~country where the law is returned for revisions to the legislature. Other options are coded 0.

To obtain the aggregate indicator of judicial review, the abovementioned data are normalized on the 0--1 interval. Consequently, the higher the aggregated measurement, the stronger the constraining power of the \fntit{de jure} constitutional review \fntit{vis-\`a-vis} legislators. For instance, in 2005, the Netherlands received 0 as there is no formal provision for the constitutional review\pozn{The Netherlands, together with Switzerland and the Scandinavian countries, are the only states in Europe which did not adopt the Kelsenian type of the Constitutional Court (Garoupa and Ginsburg 2012). For instance, article 120 of the Dutch Constitution states ``The constitutionality of Acts of Parliament and treaties shall not be reviewed by the courts'' (see, in Dutch, http://wetten.overheid.nl/BWBR0001840/geldigheidsdatum\_29-11-2012 accessed on November 29, 2012).}. Poland, in turn, receives 0.57 as the review is provided for in the Constitution. It can also be determined that the review can be pursed before and after the promulgation of the law. In addition, the list of political bodies to file the case to the Polish Constitutional Tribunal is broad.

\clearpage\newpage

It should be noted that the sample under consideration contains 24 EU countries over the period 1995--2005, for which data are easily accessible. Finland, Malta and the United Kingdom are not subject to the analysis since they do not appear in the Gutmann et al. database. Lack of comparable data on fiscal policy outcomes among the EU countries beyond 1995 downwards is a~main factor for constraining the time period under investigation. AMECO provides consistent data on the most crucial fiscal indicators for all EU countries only since 1995.\pozn{For the AMECO database, see http://ec.europa.eu/economy\_finance/ameco/user/serie/SelectSerie.cfm (accessed on October 27, 2012).} 

\fntit{Prima facie}, it seems that the main fiscal aggregate of interest, i.e. government revenue, is correlated with the degree of the \fntit{de jure} constitutional review. The Pearson's correlation coefficient\pozn{The Pearson's correlation coefficient measures the direction and the magnitude of the linear relationship between two variables.} between revenue-to-GDP ratio and degree of constitutional review is equal to roughly $-$0.54. A~more robust check for a~possible relationship between the degree of constitutional review and the fiscal policy outcomes invites, however, more advanced econometric techniques.\\[-1mm]

\podnadpis{Estimation approach}
To answer the underlying inquiry of whether the constitutional review shapes the size of government downwards, panel data model for 24 EU member states over the period 1995--2005 is applied. Panel data allows the increasing of the number of observations and exploiting the cross-time and cross-country variability of data. Specifically, the panel under consideration in this paper has a~dynamic form. It is frequent in panel studies on the size of the government to include a~lag of the dependent variable as an explanatory variable due to state dependency and persistence (see, e.g. Mukherjee 2003, Prohl and Schneider 2009). Consequently, the equation to be estimated is as follows:
\begin{eqnarray*}
y_{it}=\alpha y_{(i,t-1)} + \beta\ \mathrm{REVIEW}_{it} + \gamma' x_{it} + \eta_i + \varepsilon_{it}
\end{eqnarray*}
\vspace*{-8mm}
\begin{eqnarray*}
t=1, \ldots,T \mbox{~~and~~} i=1,\ldots,N,
\end{eqnarray*}

where $y_{it}$ stands for the size of government in country $i$~and time $t$; $y_{(i,t-1)}$ is the same fiscal policy outcome lagged by one period; $\mathrm{REVIEW}_{it}$ is a~time- and country-specific measure of the degree of the \fntit{de jure} constitutional review; $x_{it}$, is a~vector of control variables; $\eta_i$ stands for the unobserved country effects; and, lastly, $\varepsilon_{it}$ are time and country-specific error terms.

As already mentioned, the outcome variable, i.e. the size of government, is measured as government income compared to GDP (data from AMECO dataset). The key explanatory variable, i.e. $\mathrm{Review}_{it}$ is extracted from Gutmann et al. (2011). The $x_{it}$ vector contains the set of control variables. Based mostly on Persson and Tabellini (2004), the list of other explanatory variables is included in Table\;1. For summary statistics, see Table\;2.

\clearpage\newpage

\newdimen\cola
\newdimen\colb
\cola = .3\textwidth 
\colb = .65\textwidth

\popis{Table 1: Explanatory variables used in the empirical investigation}
\renewcommand{\arraystretch}{2}\small\zactab{@{}ll@{}}\hline
Variable						&	Short description 	\\\hline
\parbox{\cola}{Age dependency\\(AGE\_DEP)}		&	\parbox{\colb}{Measured as a~percentage of people above 65 years of age (own calculation based on AMECO).} \\[1mm]
Population size (POPUL)			&	\parbox{\colb}{Millions of people living in a~certain country, used in logarithmic transformation (AMECO).}\\[1mm]
Openness (TRADE)				&	\parbox{\colb}{Measured as export and import relative to GDP (own calculation based on AMECO).}\\[1mm]
\parbox{\cola}{Real income per capita\\(INCOME)}	&	\parbox{\colb}{Measured in euro (AMECO).}\\[1mm]
\parbox{\cola}{Form of government\\(PARL)}			&	\parbox{\colb}{Dummy variables for parliamentarian and presidential systems (Golder, 2000).}\\[1mm]
\parbox{\cola}{Electoral formula (PROP)}			&	\parbox{\colb}{Dummy variables for majoritarian and proportional (Golder, 2000).}\\[1mm]
Federal state (FEDER)			&	Dummy variable for federal state (Forum of Federations, 2012).\\\hline
\kontab
\zdroj{Source: own table}

At this stage of analysis three econometric methods are applied to unravel the potential relationship between the outcome and the explanatory variable of interest, i.e. ordinary least square (OLS), random effects (RE) and generalized method of moments (GMM). Especially with regard to the GMM model, one may argue that there is insufficient number of observations. As Roodman (2009) claims, GMM models are typically designed for situations with few time periods and many cross-sectional units. Excessive numbers of instruments generated by the model may lead to the lower consistency of estimators.\pozn{There are two prevailing techniques to decrease the instruments count. One is limiting the lag depth and the other is `collapsing' the instrument set (Mehrhoff, 2009). Hitherto, none of those techniques is used.} However, according to Soto (2009), in small samples with some persistency GMM estimators have lower bias and higher efficiency than OLS. Also, the application of fixed effect estimators is not recommended due to the mediocre variability in the constitutional review indicators within time.

Prior to the estimations, some basic diagnostic tests are performed. First, variance inflation factors do not identify a~multicollinearity problem\pozn{The presence of multicollinearity means that two or more independent variables are highly correlated (Wooldridge, 2009).}. Second, due to the White test indications of heteroscedasticity\pozn{The presence of heteroskedasticity means that errors do not have constant variance (Wooldridge, 2009).} of the residual variance, robust standard errors in all specifications were used. In GMM specifications, in turn, a~Windmeijer finite-sample correction is applied. Without this correction standard reported errors tend to a~downward bias (Roodman, 2006). Lastly, the Durbin-Watson test indicates the autocorrelation of errors. As already stated, the presence of the serial correlation provides the rationale for using model with a~lagged dependent variable.

\podnadpis{Results}
The coefficient estimates and their p-values for eight different specifications are presented in Table 2. The variable of interest, i.e. the \fntit{de jure} power of constitutional review, exhibits an expected sign, consistent with the hypothesis \fntit{H1} proposed in Section 4. Consequently, countries with a~high degree of constitutional review systematically tend to have smaller governments, measured as general government tax revenue compared to GDP. It seems therefore that, in countries where the judiciary is entitled to exert strong review, the ruling is in line with the public preferences about lower taxation. 

Considering the most developed model (the last column in Table 3), besides the judicial review, one can observe that other variables also have a~significant effect on revenue-to-GDP indicator. For instance, lagged revenue-to-GDP is statistically significant at the 1\% level, indicating state dependence in revenue-to-GDP. Also, trade variables are statistically significant. This would suggest that openness of the economy negatively correlates with the size of the government. Therefore, one can infer that, for EU countries, the race to the bottom argument holds\pozn{The race to the bottom hypothesis refers to a~situation where a~government decides to decrease its tax rates, especially for corporations. The creation of a~more competitive tax environment allows the attracting of more foreign investments in the face of growing globalization (Tonelson, 2002).}. 

Although those preliminary results must be treated with some reservation (they rely on strong assumptions that Courts tend to reflect public opinion), they indicate an initial support to hypothesis \fntit{H1}. While the current study presents mere correlations between the \fntit{de jure} power of judicial review and public finance outcome, the causal relationship still needs to be established. 

\begin{landscape}

\popis{Table 2: Effects of judicial review on revenue-to-GDP}
{\tabcolsep 8pt\small\begin{longtable}{|l*{7}{r}r|}\hline
				&\ce{(1)}			&\ce{(2)}			&\ce{(3)}			&\ce{(4)}			&\ce{(5)}			&\ce{(6)}			&\ce{(7)}			&\crr{(8)}		\\
VARIABLES		&\ce{OLS} 		&\ce{OLS} 		&\ce{OLS} 		&\ce{RE}		&\ce{RE}		&\ce{RE}		&\ce{GMM} 		&\crr{GMM} 	\\\hline
Lagged revenue 	&	0.942***		&	0.935*** 	&	0.932***		&	0.942***		&	0.935***		&	0.932***		&	0.551***		&	0.551***	\\
				&	(0.00)		&	(0.00)		&	(0.00)		&	(24.89647)	&	(37.45838)	&	(49.59577)	&	(0.00)		&	(0.00)	\\\hline
Review			&	-0.809*		&	-0.923**		&	-0.867*		&	-0.809*		&	-0.923**		&	-0.867*		&	-0.868***	&	-0.868***\\
				&	(0.07)		&	(0.04)		&	(0.07)		&	(0.06)		&	(0.04)		&	(0.07)		&	(0.00)		&	(0.00)	\\\hline
GDP				&	0.015		&	0.017		&	0.019		&	0.015		&	0.017		&	0.019		&	-0.135		&	-0.135	\\
				&	(0.13)		&	(0.12)		&	(0.11)		&	(0.13)		&	(0.12)		&	(0.10)		&	(0.30)		&	(0.30)	\\\hline
Trade 			&	-0.641**		&	-0.554**		&	-0.458		&	-0.641**		&	-0.554**		&	-0.458		&	-1.212		&	-1.212	\\
				&	(0.01)		&	(0.04)		&	(0.10)		&	(0.01)		&	(0.03)		&	(0.10)		&	(0.48)		&	(0.48)	\\\hline
Age dependency 	&	2.744		&	2.497		&	3.298		&	2.744		&	2.497		&	3.298		&	7.102		&	7.102	\\
				&	(0.44)		&	(0.48)		&	(0.41)		&	(0.44)		&	(0.48)		&	(0.41)		&	(0.77)		&	(0.77)	\\\hline
Log(population)	&	-0.0528		&	-0.077		&	-0.067		&	-0.058		&	-0.077		&	-0.067		&	6.893		&	6.893	\\
				&	(0.42)		&	(0.32)		&	(0.39)		&	(0.42)		&	(0.32)		&	(0.39)		&	(0.40)		&	(0.40)	\\\hline
Parliamentary 		&				&	-0.106		&	-0.136		&				&	-0.106		&	-0.136		&				&			\\
system			&				&	(0.66)		&	(0.57)		&				&	(0.66)		&	(0.57)		&				&			\\\hline
Proportional 		&				&	-0.281		&	-0.289		&				&	-0.281		&	-0.289		&				&			\\
election			&				&	(0.20)		&	(0.19)		&				&	(0.19)		&	(0.19)		&				&			\\\hline
Federation 		&				&	0.145		&	0.115		&				&	0.145		&	0.115		&				&			\\
				&				&	(0.44)		&	(0.57)		&				&	(0.44)		&	(0.57)		&				&			\\\hline
Constant 			&	2.456***		&	2.960***		&	2.995***		&	2.456***		&	2.960***		&	2.995***		&	0.041		&	0.041	\\
				&	(0.01)		&	(0.01)		&	(0.01)		&	(0.01)		&	(0.00)		&	(0.01)		&	(0.77)		&	(0.77)	\\
				&	(11.27349)	&	(17.07851)	&	(20.22370)	&	(15.00002)	&	(21.71680)	&	(27.11224)	&				&			\\\hline
Year fixed effects 	&	No			&	No			&	Yes			&	No 			&	No			&	Yes			&	No 			&	Yes		\\\hline
Observations 		&	240			&	240			&	240			&	240			&	240			&	240			&	216			&	216		\\\hline
R-squared 		&	0.964		&	0.965		&	0.967		&	-- 			&	-- 			&	-- 			&	-- 			&	-- 		\\\hline
\end{longtable}}
{\small\fntit{Note: P-values in the parentheses. *** $p<0.01$, ** $p<0.05$, * $p<0.1$}}
\zdroj{Source: own calculations}

\end{landscape}

\podnadpis{Limitations}
There are two main limitations of the empirical investigation presented in this paper. The first limitation relates to the number of observations. Currently, the estimations are based on 264 observations (panel data for 24 countries over an 11-year time period), which may lead to a~small sample bias in estimators. This reservation applies especially to methods such as the GMM, which require a~large amount of cross-sectional observations. Therefore, possible extensions of the cross-sectional dimensions could improve the reliability and robustness of the results. 

The second limitation refers to the omitted variables bias. Certain omitted variables might exert joint influence on the size of government and key variable of interest and, consequently, lead to misleading results. For instance, one could think of adding political stability and trust variables to the underlying model. The latter variable could influence both the introduction of the judicial review and the size of government. Countries with a~low level of trust could introduce a~judicial review in order to assure drafters that a~certain constitutional arrangement is kept (Ginsburg, 2006). Similarly, countries with low trust tend to have smaller governments due to people's concerns about the free riding problem linked to wide state policies, such as the welfare state (Nannestad, 2008). This might be a~serious problem if one assumes that trust is not constant over time and cannot therefore be canceled out as a~country-specific effect. Possible endogeneity problems could be mitigated through the application of instrumental variables.\pozn{To solve the current endogeneity problem, it is necessary to apply instrumental variables. As Angrist and Pischke (2008) underline, finding strong instruments is often very challenging, if not impossible.} Due to both of those limitations, i.e. sample size and endogeneity problem, the results presented in this paper should be considered as preliminary. 

\podnadpis{Conclusion}% 1970--1995, 1970--2005 zmena na ``to''
At this stage of the analysis, the preliminary results indicate that the judicial review negatively correlates with the size of government. Namely, the larger the degree of the judicial review, the smaller the revenue-to-GDP. If this conjecture factually holds, one could derive important normative implications. Hitherto, the literature is not entirely consensual about the effect of the size of government on economic growth. Most empirical studies, however, provide evidence that a~smaller size of government covariates with faster economic development. This holds especially for developed countries. For instance, Romero-\`Avila and Strauch (2008), for a~sample of 15 EU countries (old members of the EU) between 1960 and 2001, demonstrate that an increase in government size measured with revenue-to-GDP negatively affects the growth of GDP per head of population. Similar results for the EU and OECD countries between 1970 and 2004 are obtained by Afonso and Furceri (2010). On the revenue side, indirect taxes and social contributions have a~negative and statistically significant effect on growth. Also, Bergh and Karlsson (2010) find that big government negatively correlates with economic development. Their empirical investigation is pursued for OECD countries in two periods, i.e. 1970 to 1995 and 1970 to 2005. Based on those inferences, strong judicial review leading to lower government size might be indirectly conducive to faster economic growth. For this reason, policy makers should be interested in providing the judiciary with the ability to review the constitutionality of the law. 

There are certain obvious extensions and improvements to the current study. The first immediate improvement would be to increase the number of cross-sectional observations. Currently, the analysis relies on 24 EU countries between 1995 and 2005.

The second improvement would include an investigation of whether the Courts in the EU indeed echo public will and preferences in their decisions. This would allow the mitigation of the hitherto strong assumption that the Courts reflect public opinion.

Third, further empirical investigation with regard to expenditure-to-GDP, public deficit and welfare spending, could be pursued. This broader scope would allow for more accurate policy recommendations. It might be the case that, besides lower revenue-to-GDP, a~strong judicial review leads to a~higher expenditure-to-GDP or a~larger public deficit. If this is the case, the policy recommendation is not as clear cut as it is put forth above and policy makers face a~trade-off between certain fiscal policy outcomes, once a~strong judicial review is present. 

The fourth extension could be an attempt to construct more relevant \fntit{de facto} indicators of the constitutional review. As has been mentioned throughout the text, the factual influence of judicial review on public finance depends, among other things, on public support for judges' rulings, transparency surrounding judicial ruling, judicial activism and the rigidity of constitutional amendment. As studies by Feld and Voigt (2003, 2006) and also van Aaken et al. (2010) demonstrate, \fntit{de facto} indicators might have different effects compared to their \fntit{de jure} counterparts. 

Although still in a~suggestive manner, this study indicates that the judicial review by the Constitutional Court might systematically influence public finance outcomes. As shown, greater degree of constitutional review correlates with smaller size of government, measured as government income compared to GDP. These initial results invite further academic research into the topic at hand.\\[2mm]

\cast{Acknowledgements}
I am grateful to Prof. Alessio Pacces, Prof. Hans-Bernd Sch�fer, Dr. Ann-Sophie Vandenberghe and Prof. Stefan Voigt for giving comments on this paper. I~am also indebted to Jerg Gutmann, Katherine Hunt and L�szl� Pok for useful insights which were used in the current work. Special thanks, however, is devoted to Elena Kantorowicz for all her advice and constant encouragement. Any mistakes that remain are my own.\\[2mm]

\clearpage\newpage

\cast{References}
Aaken, v.\;A., Feld, L., Voigt, S. (2010). Do Independent Prosecutors Deter Political Corruption? An Empirical Evaluation across Seventy-eight Countries. \fntit{American Law and Economics Review}, 12(1), 204--244.

Afonso, A., Furceri, D. (2010). Government size, composition, volatility and economic growth. \fntit{European Journal of Political Economy}, 26(4), 517--532.

Alesina, A., Perottti, R., Tavares, J. (1998). The Political Economy of Fiscal Adjustments. \fntit{Brookings Papers on Economic Activity}, 1, 197--248.

Alesina, A., Carloni, D., Lecce, G. (2011). The Electoral Consequences of Large Fiscal Adjustments. \fntit{NBER Working Paper}, 17655.

Angrist, J., Pischke, J.-S. (2009). \fntit{Mostly Harmless Econometrics. An Empiricist's Companion}. Princeton University Press.

Bergh, A., Karlsson, M. (2010). Government size and growth: Accounting for economic freedom and globalization. \fntit{Public Choice}, 142, 195--213.

Bossuyt, M. (2008). The Belgian Constitutional Court and the re-enacting of an annulled law. \fntit{International Almanac Constitutional Justice in the New Millennium}, 200--217.

Brender, A. (2003). The Effect of Fiscal Performance on Local Government Election Results in Israel: 1989--1998. \fntit{Journal of Public Economics}, 87, 2187--2205.

Caldeira, G. (1987). Public Opinion and The U.S. Supreme Court: FDR's Court-Packing Plan. \fntit{The American Political Science Review}, 81(4), 1139--1153.

C�rdenas, M., Mej�a, C., Olivera, M. (2009). Changes in Fiscal Outcomes in Colombia: The Role of the Budget Process. In Hellerberg Mark, Scartascini Carlos and Stein Ernesto (eds.). \fntit{Who Decides the Budget? A~Political Economy Analysis of the Budget Process in Latin America}. Inter-American Development Bank.

Casillas, Ch., Enns, P., Wohlfarth, P. (2011). How Public Opinion Constrains the U.S. Supreme Court. \fntit{American Journal of Political Science}, 55(1), 74--88.

Conseil, F. (2000). \fntit{Message sur le frein \`a l'endettement Le pr�sident de la Conf�d�ration Adolf Ogi et La chanceli\`ere de la Conf�d�ration Annemarie Huber-Hotz du 5 juillet 2000}. 4295--4368.

De Vries, C., Hobolt, S. (2012). \fntit{Do Voters Blame Governments for Social Spending Cuts? Evidence from a~Natural Experiment}. http://www.sociology.ox.ac.uk/docu\-ments/epop/pa\-pers/DeVriesHobolt\_EPOP2012.pdf (accessed on October 24, 2012).

Deener, D. (1952). Judicial Review in Modern Constitutional Systems. \fntit{The American Political Science Review}, 46(4), 1079--1099.

Dotan, Y. (1998). Judicial Review and Political Accountability: The Case of the High Court of Justice in Israel. \fntit{Israel Law Review}, 32(3), 448--474.

Eslava, M. (2006). The Political Economy of Fiscal Policy: Survey. Inter-American Development Bank. Working Paper 583.

Eurobarometer (1998). Citizens and health systems: main results from a Eurobarometer survey. Employment \& social affairs.

Eurobarometer (2009). Intergenerational solidarity. Analytical report. Flash Eurobarometer 269 -- The Gallup Organisation.

\clearpage\newpage

Fallon, R. (2008). The Core of an Uneasy Case for Judicial Review. \fntit{Harvard Law Review}, 121(7), 1693--1736.

Feld, L., Voigt, S. (2003). Economic Growth and Judicial Independence: Cross Country Evidence Using a~New Set of Indicators. \fntit{European Journal of Political Economy}, 19(3), 497--527.

%\clearpage\newpage

Feld, L., Voigt, S. (2006). Judicial Independence and Economic Growth: Some Proposals Regarding the Judiciary. In Congleton Roger and Swedenborg Birgitta (eds.). \fntit{Democratic Constitutional Design and Public Policy, Analysis and Evidence}. Cambridge: MIT Press, 251--288.

Flemming, R., Wood, D. (1997). The Public and the Supreme Court: Individual Justice Responsiveness to American Policy Moods. \fntit{American Journal of Political Science}, 41(2), 468--498.

Forum of Federations (2012). \fntit{Federalism by Country}. http://www.forumfed.org/en/federa\-lism/federalismbycountry.php (accessed on October 30, 2012).

Friedman, B. (2005). The Politics of Judicial Review. \fntit{Texas Law Review}, 84(2), \mbox{257--337}.

Garlicki, L. (2007). Constitutional courts versus supreme courts. \fntit{International Journal Constitutional Law}, 5(1), 44--68.

Garoupa, N. (2011). Empirical Legal Studies and Constitutional Courts. \fntit{Indian Journal of Constitutional Law}, 5(1), 25--64.

Garoupa, N., Ginsburg, T. (2012). Building Reputation in Constitutional Courts: Party and Judicial Politics. \fntit{Arizona Journal of International and Comparative Law}, 28(3), 539--568.

Gely, R., Spiller, P. (1990). A~Rational Choice Theory of Supreme Court Statutory Decisions with Applications to the State Farm and Grove City Cases. \fntit{Journal of Law, Economics, and Organization}, 6(2), 263--300.

Gely, R., Spiller, P. (1992). The Political Economy of Supreme Court Constitutional Decisions: The Case of Roosevelt's Court-Packing Plan. \fntit{International Review of Law and Economics}, 12, 45--67. 

Giles, M., Blackstone, B., Vining, R. (2008). The Supreme Court in American Democracy: Unraveling the Linkages between Public Opinion and Judicial Decision Making. \fntit{Journal of Politics}, 70(2), 293--306.

Ginsburg, T. (2006). Economic Analysis and the Design of Constitutional Courts. \fntit{Theoretical Inquiries in Law}, 3,1.

Ginsburg, T. (2008). The Global Spread of Constitutional Review. In Whittington Keith, Kelemen Daniel, Caldeira Gregory (eds.). \fntit{Oxford Handbook of Law and Politics}, 81.

Ginsburg, T., Elkins, Z. (2009). Ancillary Powers of Constitutional Courts. \fntit{University of Texas Law Review}, 87, 1430--1461.

Golder, M. (2000). \fntit{Democratic Electoral Systems around the World, 1946--2000}. Retrieved from https://files.nyu.edu/mrg217/public/elections.html (accessed on November 17, 2012).

%\clearpage\newpage

Gutmann, J., Hayo, B., Voigt, S. (2011). Determinants of Constitutionally Safeguarded Judicial Review -- Insights Based on a~New Indicator. Working Paper. Retrieved from http://papers.ssrn.com/sol3/papers.cfm?abstract\_id=1947244 (accessed on October 23, 2012).

\clearpage\newpage

Gwartney, J., Lawson, R., Hall, J. (2012). \fntit{Economic Freedom of the World: 2012 Annual Report}. Fraser Institute.

%\clearpage\newpage

Von Hagen, J. (1992). \fntit{Budgeting Procedures and Fiscal Performance in the European Communities}. Commission of the European Communities. DG ECFIN. European Economy Paper No. 96.

Hansen, J.\;M. (1998). Individuals, Institutions, and Public Preferences over Public Finance. \fntit{American Political Science Review}, 92(3), 513--531.

Hayo, B., Voigt, S. (2007). Explaining de facto judicial independence. \fntit{International Review of Law and Economics}, 27, 269--290.

Henisz, W. (2000). The Institutional Environment for Economic Growth. \fntit{Economics and Politics}, 12(1), 1--31.

Jolls, Ch., Sunstein, C., Thaler, R. (1998). A~Behavioral Approach to Law and Economics. \fntit{Stanford Law Review}, 50, 1471--1547.

Kelsen, H. (1942). Judicial Review of Legislation. \fntit{The Journal of Politics}, 4(2), 183--200.

Kirchg�ssner, G. (2005). Sustainable Fiscal Policy in a~Federal State: The Swiss Example. \fntit{Swiss Political Science Review}, 11(4), 19--46.

Landes, W., Posner R. (1975). The Independent Judiciary in an Interest-Group Perspective. \fntit{Journal of Law and Economics}, 18(3). Economic Analysis of Political Behavior: Universities-National Bureau Conference Series Number 29. 875--901.

Link, M. (1995). Tracking Public Mood in the Supreme Court: Cross-Time Analyses of Criminal Procedure and Civil Rights Cases. \fntit{Political Research Quarterly}, 48(1), 61--78.

McGuire, K., Stimson, J. (2004). The Least Dangerous Branch Revisited: New Evidence on Supreme Court Responsiveness to Public Preferences. \fntit{The Journal of Politics}, 66(4), 1018--1035.

Mehrhoff, J. (2009). A~solution to the problem of too many instruments in dynamic panel data GMM. Deutsche Bundesbank. Discussion Paper Series 1: Economic Studies. No 31.

Mishler, W., Sheehan, R. (1993). The Supreme Court as a Countermajoritarian Institution? The Impact of Public Opinion on Supreme Court Decisions. \fntit{The American Political Science Review}, 87(1), 87--101. 

Mishler, W., Sheehan, R. (1994). Popular Influence on Supreme Court Decisions. \fntit{The American Political Science Review}, 88(3), 716--24.

Mishler, W., Sheehan, R. (1996). Public Opinion, the Attitudinal Model, and Supreme Court Decision Making: A~Micro-Analytic Perspective. \fntit{Journal of Politics}, 58(1), \mbox{169--200}.

Mueller, D. (2003). \fntit{Public Choice III}. Cambridge University Press.

%\clearpage\newpage

Mukherjee, B. (2003). Political Parties and the Size of Government in Multiparty Legislatures. Examining Cross-Country and Panel Data Evidence. \fntit{Comparative Political Studies}, 36(6), 699--728.

Nannestad, P. (2008). What Have We Learned About Generalized Trust, If Anything? \fntit{Annual Review of Political Science}, 11, 413--437.

Padovano, F., Sgarra, G., Fiorino, N. (2003). Judicial Branch, Checks and Balances and Political Accountability. \fntit{Constitutional Political Economy}, 14, 47--79.

%\clearpage\newpage

Peltzman, S. (1992). Voters as Fiscal Conservatives. \fntit{The Quarterly Journal of Economics}, 107(2), 327--361.

Persson, T., Tabellini, G. (2003). \fntit{Economic Effects of Constitutions}. Cambridge: MIT Press.

Persson, T., Tabellini, G. (2004). Constitutional Rules and Fiscal Policy Outcomes. \fntit{The American Economic Review}, 94(1), 25--45.

%\clearpage\newpage

Pommerehne, W., Schneider, F. (1978). Fiscal Illusion, Political Institutions, and Local Public Spending. \fntit{Kyklos}, 31(3), 381--408.

Posner, R. (2008). \fntit{How Judges Think}. Harvard University Press.

Prohl, S., Schneider, F. (2009). Does Decentralization Reduce Government Size? A~Qualitative Study of the Decentralization Hypothesis. \fntit{Public Finance Review}, 37(6), 639--664.

Raudla, R. (2010). \fntit{Constitution, Public Finance, and Transition. Theoretical Developments in Constitutional Public Finance and the Case of Estonia}. Finanzsoziologie 4, Frankfurt am Main: Peter Lang.

Raudla, R. (2011). Effects of a~Constitution on Taxation: The Role of Constitutional Review in the Development of Tax Laws in Estonia. \fntit{Halduskultuur -- Administrative Culture}, 12(1), 76--105.

Romero-\`Avila, D., Strauch, R. (2008). Public finances and long-term growth in Europe: Evidence from a~panel data analysis. \fntit{European Journal of Political Economy}, 24(1), 172--191.

Roodman, D. (2006). How to Do xtabond2: An Introduction to ``Difference'' and ``System'' GMM in Stata. Center for Global Development. Working Paper Number 103.

Roodman, D. (2009). Practitioners' Corner. A~Note on the Theme of Too Many Instruments. \fntit{Oxford Bulletin of Economics and Statistics}, 71(1).

Sadurski, W. (2002). \fntit{Constitutional Justice, East and West. Democratic Legitimacy and Constitutional Courts in Post-Communist Europe in A~Comparative Perspective}. Kluwer Law International. 

Schauer, F. (2012). The Political Risks (if any) of Breaking the Law. \fntit{Journal of Legal Analysis}, 4(1), 83--101.

Sch�fer, H.-B., Ott, C. (2004). \fntit{The Economic Analysis of Civil Law}. Edward Elgar Publishing. 

Schuknecht, L. (1994). Political Business Cycles and Expenditure Policies in Developing Countries. IMF Working Paper 121. 

Schwartz, H. (2002). \fntit{The struggle for constitutional justice in post-communist Europe}. Chicago: University of Chicago Press.

%\clearpage\newpage

Segal, J., Spaeth, H. (2002). \fntit{The Supreme Court and the Attitudinal Model Revisited}. Cambridge University Press.

Soto, M. (2009). System GMM estimation with a~small sample. Barcelona Economics Working Paper Series Working Paper no. 395.

Stimson, J., MacKuen, M., Erikson, R. (1995). Dynamic Representation. \fntit{American Political Science Review}, 89(3), 543--65.

%\clearpage\newpage

Stone, A. (1995). Coordinate Construction in France and Germany. In Tate Neal and Vallinder Torbj�rn (eds.). \fntit{The Global Expansion of Judicial Power}. New York University Press: New York, London. 205--229.

\clearpage\newpage

Stone Sweet, A. (2000). \fntit{Governing with Judges. Constitutional Politics in Europe}. Oxford University Press.

Stone Sweet, A. (2007). The politics of constitutional review in France and Europe. \fntit{International Journal of Constitutional Review}, 5(1), 69--92.

Tonelson, A. (2002). \fntit{The Race to the Bottom. Why a~Worldwide Worker Surplus and Uncontrolled Free Trade are Sinking American Living Standards}. Boulder: Westview Press.

Tridimas, G. (2005). Judges and Taxes: Judicial review, judicial independence and the size of government. \fntit{Constitutional Political Economy}, 16, 5--30.

Trybuna� K. (2003). Prawo podatkowe w �wietle orzecznictwa Trybuna�u Konstytucyjnego w 2002 r. Wydawnictwo Trybuna�u Konstytucyjnego. 15--27. 

Tsebelis, G. (2002). \fntit{Veto Players: How Political Institutions Work}. Princeton University Press.

Tushnet, M. (2010). How Different are Waldron's and Fallon's Core Cases for and against Judicial Review? \fntit{Oxford Journal of Legal Studies}, 30(1), 49--70.

Ura, J., Wohlfarth, P. (2010). ``An Appeal to the People'': Public Opinion and Congressional Support for the Supreme Court. \fntit{The Journal of Politics}, 72(4), 939--956.

Vallinder, T. (1995). When the Courts Go Marching In. In Tate Neal and Vallinder Torbj�rn (eds.). \fntit{The Global Expansion of Judicial Power}. New York University Press: New York, London. 13--26.

Vanberg, G. (2005). \fntit{The Politics of Constitutional Review in Germany}. Cambridge: Cambridge University Press. 

Vaubel, R. (1996). Constitutional Safeguards Against Centralization in Federal States: An International Cross-Section Analysis. \fntit{Constitutional Political Economy}, 7, 79--102.

Vaubel, R. (2009). Constitutional courts as promoters of political centralization: lessons for the European Court of Justice. \fntit{European Journal of Law and Economics}, 28, 203--222.

Volcansek, M. (2000). \fntit{Constitutional Politics in Italy: The Constitutional Court}. Houndsmills, Basingstoke, UK: Macmillan Press and New York: St. Martin's Press.

Volcansek, M. (2001). Constitutional courts as veto players: Divorce and decrees in Italy. \fntit{European Journal of Political Research}, 39, 347--372.

Wagner, R. (1976). Revenue Structure, Fiscal Illusion, and Budgetary Choice. \fntit{Public Choice}, 25, 45--61.

%\clearpage\newpage

Waldron, J. (2006). The Core of the Case Against Judicial Review. \fntit{The Yale Law Review}, 115, 1346--1406.

Welch, S. (1985). The ``More for Less'' Paradox: Public Attitudes on Taxing and Spending. \fntit{Public Opinion Quarterly}, 46(3).

Wittrup, J. (2010). Budgeting in the Era of Judicial Independence. International Journal For Court Administration. April.

Wooldridge, J. (2009). \fntit{Introductory Econometrics: A~Modern Approach}. South-Western Cengage Learning.
