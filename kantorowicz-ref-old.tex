\cast{References}
Aaken van Anne, Feld Lars, Voigt Stefan (2010). Do Independent Prosecutors Deter Political Corruption? An Empirical Evaluation across Seventy-eight Countries. \fntit{American Law and Economics Review}, 12(1), 204--244.

Afonso, Ant�nio, Furceri, Davide (2010). Government size, composition, volatility and economic growth. \fntit{European Journal of Political Economy}, 26(4), 517--532.

Alesina, Alberto, Perottti, Roberto, Tavares Jos� (1998). The Political Economy of Fiscal Adjustments. \fntit{Brookings Papers on Economic Activity}, 1, 197--248.

Alesina, Alberto, Carloni, Dorian, Lecce, Giampaolo (2011). The Electoral Consequences of Large Fiscal Adjustments. NBER Working Paper No. 17655.

Angrist, Joshua, Pischke, J�rn-Steffen (2009). \fntit{Mostly Harmless Econometrics. An Empiricist's Companion}. Princeton University Press.

Bergh, Andreas, Karlsson, Martin (2010). Government size and growth: Accounting for economic freedom and globalization. \fntit{Public Choice}, 142, 195--213.

Bossuyt, Marc (2008). The Belgian Constitutional Court and the re-enacting of an annulled law. \fntit{International Almanac Constitutional Justice in the New Millennium}, 200--217.

Brender, Adi (2003). The Effect of Fiscal Performance on Local Government Election Results in Israel: 1989--1998. \fntit{Journal of Public Economics}, 87, 2187--2205.

Caldeira, Gregory (1987). Public Opinion and The U.S. Supreme Court: FDR's Court-Packing Plan. \fntit{The American Political Science Review}, 81(4), 1139--1153.

C�rdenas, Mauricio, Mej�a, Carolina, Olivera, Mauricio (2009). Changes in Fiscal Outcomes in Colombia: The Role of the Budget Process. In Hellerberg Mark, Scartascini Carlos and Stein Ernesto (eds.). \fntit{Who Decides the Budget? A~Political Economy Analysis of the Budget Process in Latin America}. Inter-American Development Bank.

Casillas, Christopher, Enns, Peter, Wohlfarth, Patrick (2011). How Public Opinion Constrains the U.S. Supreme Court. \fntit{American Journal of Political Science}, 55(1), 74--88.

Conseil, Federal (2000). \fntit{Message sur le frein \`a l'endettement Le pr�sident de la Conf�d�ration Adolf Ogi et La chanceli\`ere de la Conf�d�ration Annemarie Huber-Hotz du 5 juillet 2000}. 4295--4368.

De Vries, Catherine, Hobolt, Sara (2012). \fntit{Do Voters Blame Governments for Social Spending Cuts? Evidence from a~Natural Experiment}. http://www.sociology.ox.ac.uk/docu\-ments/epop/papers/DeVriesHobolt\_EPOP2012.pdf (accessed on October 24, 2012).

Deener, David (1952). Judicial Review in Modern Constitutional Systems. \fntit{The American Political Science Review}, 46(4), 1079--1099.

Dotan, Yoav (1998). Judicial Review and Political Accountability: The Case of the High Court of Justice in Israel. \fntit{Israel Law Review}, 32(3), 448--474.

Eslava, Marcela (2006). The Political Economy of Fiscal Policy: Survey. Inter-American Development Bank. Working Paper 583.

Eurobarometer (1998). Citizens and health systems: main results from a Eurobarometer survey. Employment \& social affairs.

Eurobarometer (2009). Intergenerational solidarity. Analytical report. Flash Eurobarometer 269 -- The Gallup Organisation.

\clearpage\newpage

Fallon, Richard (2008). The Core of an Uneasy Case for Judicial Review. \fntit{Harvard Law Review}, 121(7), 1693--1736.

Feld, Lars, Voigt, Stefan (2003). Economic Growth and Judicial Independence: Cross Country Evidence Using a~New Set of Indicators. \fntit{European Journal of Political Economy}, 19(3), 497--527.

%\clearpage\newpage

Feld, Lars, Voigt, Stefan (2006). Judicial Independence and Economic Growth: Some Proposals Regarding the Judiciary. In Congleton Roger and Swedenborg Birgitta (eds.). \fntit{Democratic Constitutional Design and Public Policy, Analysis and Evidence}. Cambridge: MIT Press, 251--288.

Flemming, Roy, Wood, Dan (1997). The Public and the Supreme Court: Individual Justice Responsiveness to American Policy Moods. \fntit{American Journal of Political Science}, 41(2), 468--498.

Forum of Federations (2012). \fntit{Federalism by Country}. http://www.forumfed.org/en/federa\-lism/federalismbycountry.php (accessed on October 30, 2012).

Friedman, Barry (2005). The Politics of Judicial Review. \fntit{Texas Law Review}, 84(2), \mbox{257--337}.

Garlicki, Lech (2007). Constitutional courts versus supreme courts. \fntit{International Journal Constitutional Law}, 5(1), 44--68.

Garoupa, Nuno (2011). Empirical Legal Studies and Constitutional Courts. \fntit{Indian Journal of Constitutional Law}, 5(1), 25--64.

Garoupa, Nuno, Ginsburg, Tom (2012). Building Reputation in Constitutional Courts: Party and Judicial Politics. \fntit{Arizona Journal of International and Comparative Law}, 28(3), 539--568.

Gely, Rafael, Spiller, Pablo (1990). A~Rational Choice Theory of Supreme Court Statutory Decisions with Applications to the State Farm and Grove City Cases. \fntit{Journal of Law, Economics, and Organization}, 6(2), 263--300.

Gely, Rafael, Spiller, Pablo (1992). The Political Economy of Supreme Court Constitutional Decisions: The Case of Roosevelt's Court-Packing Plan. \fntit{International Review of Law and Economics}, 12, 45--67. 

Giles, Michael, Blackstone, Bethany, Vining Richard (2008). The Supreme Court in American Democracy: Unraveling the Linkages between Public Opinion and Judicial Decision Making. \fntit{Journal of Politics}, 70(2), 293--306.

Ginsburg, Tom (2006). Economic Analysis and the Design of Constitutional Courts. \fntit{Theoretical Inquiries in Law}, 3,1.

Ginsburg, Tom (2008). The Global Spread of Constitutional Review. In Whittington Keith, Kelemen Daniel, Caldeira Gregory (eds.). \fntit{Oxford Handbook of Law and Politics}, 81.

Ginsburg, Tom, Elkins, Zachary (2009). Ancillary Powers of Constitutional Courts. \fntit{University of Texas Law Review}, 87, 1430--1461.

Golder, Matt (2000). \fntit{Democratic Electoral Systems around the World, 1946--2000}. Retrieved from https://files.nyu.edu/mrg217/public/elections.html (accessed on November 17, 2012).

\clearpage\newpage

Gutmann, Jerg, Hayo, Bernd, Voigt, Stefan (2011). Determinants of Constitutionally Safeguarded Judicial Review -- Insights Based on a~New Indicator. Working Paper. Retrieved from http://papers.ssrn.com/sol3/papers.cfm?abstract\_id=1947244 (accessed on October 23, 2012).

Gwartney, James, Lawson, Robert, Hall, Joshua (2012). \fntit{Economic Freedom of the World: 2012 Annual Report}. Fraser Institute.

%\clearpage\newpage

Von Hagen, J�rgen (1992). \fntit{Budgeting Procedures and Fiscal Performance in the European Communities}. Commission of the European Communities. DG ECFIN. European Economy Paper No. 96.

Hansen, John Mark (1998). Individuals, Institutions, and Public Preferences over Public Finance. \fntit{American Political Science Review}, 92(3), 513--531.

Hayo, Bernd, Voigt, Stefan (2007). Explaining de facto judicial independence. \fntit{International Review of Law and Economics}, 27, 269--290.

Henisz, Witold (2000). The Institutional Environment for Economic Growth. \fntit{Economics and Politics}, 12(1), 1--31.

Jolls, Christine, Sunstein, Cass, Thaler, Richard (1998). A~Behavioral Approach to Law and Economics. \fntit{Stanford Law Review}, 50, 1471--1547.

Kelsen, Hans (1942). Judicial Review of Legislation. \fntit{The Journal of Politics}, 4(2), 183--200.

Kirchg�ssner, Gebhard (2005). Sustainable Fiscal Policy in a~Federal State: The Swiss Example. \fntit{Swiss Political Science Review}, 11(4), 19--46.

Landes, Wiliam, Posner Richard (1975). The Independent Judiciary in an Interest-Group Perspective. \fntit{Journal of Law and Economics}, 18(3). Economic Analysis of Political Behavior: Universities-National Bureau Conference Series Number 29. 875--901.

Link, Michael (1995). Tracking Public Mood in the Supreme Court: Cross-Time Analyses of Criminal Procedure and Civil Rights Cases. \fntit{Political Research Quarterly}, 48(1), 61--78.

McGuire, Kevin, Stimson, James (2004). The Least Dangerous Branch Revisited: New Evidence on Supreme Court Responsiveness to Public Preferences. \fntit{The Journal of Politics}, 66(4), 1018--1035.

Mehrhoff, Jens (2009). A~solution to the problem of too many instruments in dynamic panel data GMM. Deutsche Bundesbank. Discussion Paper Series 1: Economic Studies. No 31.

Mishler, William, Sheehan, Reginald (1993). The Supreme Court as a Countermajoritarian Institution? The Impact of Public Opinion on Supreme Court Decisions. \fntit{The American Political Science Review}, 87(1), 87--101. 

Mishler, William, Sheehan, Reginald (1994). Popular Influence on Supreme Court Decisions. \fntit{The American Political Science Review}, 88(3), 716--24.

Mishler, William, Sheehan, Reginald (1996). Public Opinion, the Attitudinal Model, and Supreme Court Decision Making: A~Micro-Analytic Perspective. \fntit{Journal of Politics}, 58(1), 169--200.

Mueller, Dennis (2003). \fntit{Public Choice III}. Cambridge University Press.

\clearpage\newpage

Mukherjee, Bumba (2003). Political Parties and the Size of Government in Multiparty Legislatures. Examining Cross-Country and Panel Data Evidence. \fntit{Comparative Political Studies}, 36(6), 699--728.

Nannestad, Peter (2008). What Have We Learned About Generalized Trust, If Anything? \fntit{Annual Review of Political Science}, 11, 413--437.

Padovano, Fabio, Sgarra, Grazia, Fiorino, Nadia (2003). Judicial Branch, Checks and Balances and Political Accountability. \fntit{Constitutional Political Economy}, 14, 47--79.

%\clearpage\newpage

Peltzman, Sam (1992). Voters as Fiscal Conservatives. \fntit{The Quarterly Journal of Economics}, 107(2), 327--361.

Persson, Torsten, Tabellini, Guido (2003). \fntit{Economic Effects of Constitutions}. Cambridge: MIT Press.

Persson, Torsten, Tabellini, Guido (2004). Constitutional Rules and Fiscal Policy Outcomes. \fntit{The American Economic Review}, 94(1), 25--45.

Pommerehne, Werner, Schneider, Friedrich (1978). Fiscal Illusion, Political Institutions, and Local Public Spending. \fntit{Kyklos}, 31(3), 381--408.

Posner, Richard (2008). \fntit{How Judges Think}. Harvard University Press.

Prohl, Silika, Schneider, Friedrich (2009). Does Decentralization Reduce Government Size? A~Qualitative Study of the Decentralization Hypothesis. \fntit{Public Finance Review}, 37(6), 639--664.

Raudla, Ringa (2010). \fntit{Constitution, Public Finance, and Transition. Theoretical Developments in Constitutional Public Finance and the Case of Estonia}. Finanzsoziologie 4, Frankfurt am Main: Peter Lang.

Raudla, Ringa (2011). Effects of a~Constitution on Taxation: The Role of Constitutional Review in the Development of Tax Laws in Estonia. \fntit{Halduskultuur -- Administrative Culture}, 12(1), 76--105.

Romero-\`Avila, Diego, Strauch, Rolf (2008). Public finances and long-term growth in Europe: Evidence from a~panel data analysis. \fntit{European Journal of Political Economy}, 24(1), 172--191.

Roodman, David (2006). How to Do xtabond2: An Introduction to ``Difference'' and ``System'' GMM in Stata. Center for Global Development. Working Paper Number 103.

Roodman, David (2009). Practitioners' Corner. A~Note on the Theme of Too Many Instruments. \fntit{Oxford Bulletin of Economics and Statistics}, 71(1).

Sadurski, Wojciech (2002). \fntit{Constitutional Justice, East and West. Democratic Legitimacy and Constitutional Courts in Post-Communist Europe in A~Comparative Perspective}. Kluwer Law International. 

Schauer, Frederick (2012). The Political Risks (if any) of Breaking the Law. \fntit{Journal of Legal Analysis}, 4(1), 83--101.

Sch�fer, Hans-Bernd, Ott, Claus (2004). \fntit{The Economic Analysis of Civil Law}. Edward Elgar Publishing. 

Schuknecht, Ludger (1994). Political Business Cycles and Expenditure Policies in Developing Countries. IMF Working Paper 121. 

Schwartz, Herman (2002). \fntit{The struggle for constitutional justice in post-communist Europe}. Chicago: University of Chicago Press.

\clearpage\newpage

Segal, Jeffrey, Spaeth, Harold (2002). \fntit{The Supreme Court and the Attitudinal Model Revisited}. Cambridge University Press.

Soto, Marcelo (2009). System GMM estimation with a~small sample. Barcelona Economics Working Paper Series Working Paper no. 395.

Stimson, James, MacKuen, Michael, Erikson, Robert (1995). Dynamic Representation. \fntit{American Political Science Review}, 89(3), 543--65.

%\clearpage\newpage

Stone, Alec (1995). Coordinate Construction in France and Germany. In Tate Neal and Vallinder Torbj�rn (eds.). \fntit{The Global Expansion of Judicial Power}. New York University Press: New York, London. 205--229.

Stone Sweet, Alec (2000). \fntit{Governing with Judges. Constitutional Politics in Europe}. Oxford University Press.

Stone Sweet, Alec (2007). The politics of constitutional review in France and Europe. \fntit{International Journal of Constitutional Review}, 5(1), 69--92.

Tonelson, Alan (2002). \fntit{The Race to the Bottom. Why a~Worldwide Worker Surplus and Uncontrolled Free Trade are Sinking American Living Standards}. Boulder: Westview Press.

Tridimas, George (2005). Judges and Taxes: Judicial review, judicial independence and the size of government. \fntit{Constitutional Political Economy}, 16, 5--30.

Trybuna? Konstytucyjny (2003). Prawo podatkowe w ?wietle orzecznictwa Trybuna?u Konstytucyjnego w 2002 r. Wydawnictwo Trybuna?u Konstytucyjnego. 15--27. 

Tsebelis, George (2002). \fntit{Veto Players: How Political Institutions Work}. Princeton University Press.

Tushnet, Mark (2010). How Different are Waldron's and Fallon's Core Cases for and against Judicial Review? \fntit{Oxford Journal of Legal Studies}, 30(1), 49--70.

Ura, Joseph, Wohlfarth, Patrick (2010). ``An Appeal to the People'': Public Opinion and Congressional Support for the Supreme Court. \fntit{The Journal of Politics}, 72(4), 939--956.

Vallinder, Torbj�rn (1995). When the Courts Go Marching In. In Tate Neal and Vallinder Torbj�rn (eds.). \fntit{The Global Expansion of Judicial Power}. New York University Press: New York, London. 13--26.

Vanberg, Georg (2005). \fntit{The Politics of Constitutional Review in Germany}. Cambridge: Cambridge University Press. 

Vaubel, Roland (1996). Constitutional Safeguards Against Centralization in Federal States: An International Cross-Section Analysis. \fntit{Constitutional Political Economy}, 7, 79--102.

Vaubel, Roland (2009). Constitutional courts as promoters of political centralization: lessons for the European Court of Justice. \fntit{European Journal of Law and Economics}, 28, 203--222.

Volcansek, Mary (2000). \fntit{Constitutional Politics in Italy: The Constitutional Court}. Houndsmills, Basingstoke, UK: Macmillan Press and New York: St. Martin's Press.

Volcansek, Mary (2001). Constitutional courts as veto players: Divorce and decrees in Italy. \fntit{European Journal of Political Research}, 39, 347--372.

Wagner, Richard (1976). Revenue Structure, Fiscal Illusion, and Budgetary Choice. \fntit{Public Choice}, 25, 45--61.

\clearpage\newpage

Waldron, Jeremy (2006). The Core of the Case Against Judicial Review. \fntit{The Yale Law Review}, 115, 1346--1406.

Welch, Susan (1985). The ``More for Less'' Paradox: Public Attitudes on Taxing and Spending. \fntit{Public Opinion Quarterly}, 46(3).

Wittrup, Jesper (2010). Budgeting in the Era of Judicial Independence. International Journal For Court Administration. April.

Wooldridge, Jeffrey (2009). \fntit{Introductory Econometrics: A~Modern Approach}. South-Western Cengage Learning.