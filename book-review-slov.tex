\newpage\nuluj
\pagestyle{unionlaw}

\autorb{RASTISLAV FUNTA, �TEFAN NEBESK�, FILIP JURI�}
\nadpisb{Pr�vo Eur�pskej �nie (European Union Law)}
\vyd{Tribun EU Brno, 2014, 546 pages}{ISBN 978-80-263-0565-1}
\uvod{union-law-gb.jpg}

New to market for 2014 is a~monograph entitled ``European Union Law" by a~young team of authors led by JUDr. Rastislav Funta, Ph.D., LL.M. and his two colleagues JUDr. �tefan Nebesk�, Ph.D., M.E.S. and Mgr. Filip Juri�. Although this is the work of a~young team of authors, it should be noted that it is an experienced team. The lead author has gained, in addition to theoretical experience,  practical vocational skills thanks to  activity in the German media group Bartelsmann, an Italian consultancy company, an international IT company, cooperation with the College of Europe$/$the European Commission, publishing and lecturing activities, and activities in the field of political lobbying. Another of the authors works for the Legal Service of the European Banking Authority in London and the third co-author is currently active in the field of legal advice, with a~focus on M\&A and transaction advisory.

The system of law created by the European Union (EU) is now a~solid component of our political and social reality. Thousands of decisions are accepted annually on the basis of EU agreements and these significantly help to create the reality which influences Member States of the EU and its citizens. Citizens of the individual states have ceased to be only citizens of their country, their towns or municipalities, and have also become European Union citizens. It is therefore very important that they should be kept informed about the legal system which influences their lives every day. However, the structure of the EU and its system of law is not always clear to its citizens. This is often caused by the amended contracts themselves, which are very often confused and can only be understood with great difficulty. It is thus important to identify irregularities in the many terms with which these contracts attempt to react to new facts. It is also vital to attempt to make the structure of the EU and the pillars of the European system of law clear, thereby contributing to better understanding on the part of EU citizens. 

On the one hand, rapid integration within the EU brought about incentives for growth, but on the other hand, it also created vulnerabilities that were, in many EU Member States, aggravated by the recession. Overcoming the consequences of the global crisis is only possible via processes of adaptation that respond to its multidimensional causation. Europe's mission in the 21st century is primarily to ensure peace, prosperity and stability, support balanced economic and social development, face the challenges of globalization and preserve the diversity of the people of Europe and uphold the values common to all Europeans (sustainable development, respect for human rights, etc.). Globalization keeps bringing new challenges. The EU will need a~wide range of competencies to be able to adapt flexibly to a~rapidly changing and highly interconnected world. The Treaty of Lisbon, signed on 13 December, 2007, aims to provide the Union with the legal framework and tools necessary to meet these future challenges and fulfill the requirements of EU citizens. With it, the Union can better focus on policy challenges, including globalization and climate change. The wealth and success of Europe lies in its diversity and a~fair balance between the interests of larger and smaller countries.

The European Union, as was mentioned in Judgment C-294/83, Les Verts$/$Parlament, is ``the community of law''. The EU differs from the classical international community of states in that the Member States have given up parts of their sovereignty in favour of the EU and they have been provided by the Union with their own rights independent from the Member States. It is necessary to mention that the EU does not have absolute power. In the exercise of its powers, the EU can issue sovereign European acts, which are, in terms of effectiveness, the same as legal acts by Member States. With the Maastricht Treaty, we have entered a~new stage on the path to the political unification of Europe and begun the process of forming a~closer alliance between the European nations. Other developments which have been achieved by the EU are thanks to the Amsterdam Treaty and the Treaty of Nice. While the goal of the first treaty was mainly the creation of institutional and political conditions in order to meet new challenges, the goal of the Treaty of Nice has been the increasing of the effectiveness and legitimacy of Community institutions, as well as preparation for the expansion of  the EU.

A unified Europe has been built in accordance with the core concepts of values (Article\;2 of the EU Treaty) by which the Member States feel bound, with the application of these values entrusted to EU executive institutions. These basic values are: securing permanent peace, unity, equality, freedom, solidarity and safety. The EU explicitly claims to respect the principles of democracy and accord a~common legal status to all Member States, while also protecting fundamental human rights. These values are at the same time a~foundation stone for those states which would in future like to join the EU. If a~Member State seriously and persistently violates these values and rules, penalties are inevitable. 

No EU citizen should be at a~disadvantage, i.e. discriminated against on the basis of nationality. Moreover, all citizens of the Union are equal before the law. With regard to the Member States, the principle of equality (Chapter III Equality, the Charter of Fundamental Rights of the European Union) means that none is favored over another and that intrinsic differences such as size, population of the country or different structures can be resolved only in accordance with the principle of equality. 

Solidarity (Chapter IV Solidarity, Charter of Fundamental Rights of the European Union) is also a~necessary correctional freedom, because the unscrupulous exploitation of freedom is always carried out at the expense of others. With traditional international organizations, the EU has in common only that it also forms the basis for international agreements, but the EU has become far removed from these international roots. The founding acts of the EU, which are based on international agreements, led to the establishment of a~separate EU with its own sovereign rights and powers. Member States surrendered part of their sovereignty in favor of the Union. 

The EU's role in economic policy is not to set and enforce European economic policy, but to coordinate the economic policies of the Member States (Article 121 TFEU) so that the economic decisions of one or more Member States do not impair the functioning of the internal market. For this purpose, the Stability and Growth Pact sets out detailed criteria according to which each Member State must make decisions on fiscal policy. If these criteria cannot be met, the Member State may give notice to the European Commission (Article 121 (4) TFEU) and the EU Council may impose excessive budget deficit procedures or impose sanctions. 

In addition to economic and monetary policy, the EU plays a~role in a~number of other areas of economic policy. These include in particular agricultural policy and fisheries policy, transport policy, consumer policy, structural and cohesion policy, research policy and development policy on space, environmental policy, health policy, trade policy and energy policy. In the area of social policy, the EU must ensure that the benefits of economic integration not only apply to economically active subjects, but that they respect the social dimension of the internal market.

We should not forget that the Union has an institutional framework which aims to promote its values (Article 2 TEU), to achieve its objectives (Article 3 TEU), serve its interests, those of its citizens and the interests of the Member States, and ensure coherence, effectiveness and continuity in its policies and actions. The EU has an entire institutional system that allows it to provide unified Europe with new impulses and objectives and to areas falling within its powers create rights which are equally fair for all Member States. Even with all the imperfections that characterize EU law, the contribution made by the EU legal system when dealing with the political, economic and social problems of the EU Member States is priceless. 

History confirms that the creation of the European Union was a~key element in bringing about improvements in peace and security in anticipation of future economic development in the EU. However, there are new challenges (the ongoing crisis, uncertainty and instability in financial markets) to which it must respond.

We hope that this publication will help to explain to the general public the essence of the EU and its importance. However, its intended audience is law schools, professionals in the field of European law, economists and managers, as well as other professionals in related fields, such as political science or European Studies. 
